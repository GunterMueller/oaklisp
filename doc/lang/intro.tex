% This file is part of Oaklisp.
%
% This program is free software; you can redistribute it and/or modify
% it under the terms of the GNU General Public License as published by
% the Free Software Foundation; either version 2 of the License, or
% (at your option) any later version.
%
% This program is distributed in the hope that it will be useful,
% but WITHOUT ANY WARRANTY; without even the implied warranty of
% MERCHANTABILITY or FITNESS FOR A PARTICULAR PURPOSE.  See the
% GNU General Public License for more details.
%
% The GNU GPL is available at http://www.gnu.org/licenses/gpl.html
% or from the Free Software Foundation, 59 Temple Place - Suite 330,
% Boston, MA 02111-1307, USA


\chapter{Introduction}

\emph{This is the introduction to the original Oaklisp proposal which
we wrote in January 1985.  Although the core language hasn't changed
since then, some of the periperal ideas in the proposal have been
modified or abandoned.}

One of the most interesting language ideas to emerge from the 1970's
was the object-oriented programming model.  Although this model has
been incorporated to some extent in a number of recent Lisps, these
implementations have not had the generality and power that
characterize a true object-based system like Smalltalk.  The most
significant trend in the contemporary Lisp world is the move toward
lexical scoping, which was initiated by Steele and Sussman with their
Scheme papers and continued most faithfully by the designers of T.

The major goal of Oaklisp was to combine the ideas of Smalltalk and
Scheme in a simple but expressive language that inherits their
exemplary properties of modularity and consistency.  Unlike T, which
adds an object-based capability to Scheme by constructing objects out
of closures, Oaklisp builds a lexically scoped Lisp system on top of a
general message-passing system that allows for full inheritance of
methods from multiple superclasses.

The first design choice for Oaklisp was the extent to which we should
push the object-based model.  We agreed with the designers of ADA that
the packaging of state and procedures into new types can be a
significant modularity tool for users of the language.  We also agreed
with the designers of the Lisp Machine that single-user Lisp systems
should be open, with no clear line between system and user code.
Therefore, both to allow the system implementors to use powerful
user-level tools and to allow users to easily manipulate the system,
we decided that absolutely everything should be a full-fledged object
in Oaklisp.  There is no reason why user-defined types should be any
different from the types used to build the system.  Oaklisp stands in
marked contrast to other object-oriented Lisp systems which have magic
data types that are not part of the user-level type
hierarchy.\footnote{For example, ZetaLisp flavors are not themselves
instances of flavors.} Our current Oaklisp implementation takes this
idea to such an extreme that there are no magic objects anywhere in
the system, no matter how deep you go.  This meant not only that the
vast majority of the system could be written in Lisp, but that the
construction of the debugger and garbage collector was greatly
simplified.

As a corollary to the previous decision, we decided that all
computation should be performed by methods that are invoked after a
search up the type hierarchy.  Functions can be thought of as methods
attached to the top of the hierarchy, since they are methods that can
perform an operation on any type.  This leads to the interesting
result that after a function has been defined, a new method can be
added to take over that operation for a special case.

The power of the method-invocation model of computation is derived
from the generality of the inheritance mechanism.\footnote{It is
primarily the lack of inheritance that weakens the T object facility.}
A simple type-tree would have provided Oaklisp with the ability to
define shadowable system-wide defaults for \df{print} and so forth.
However, we felt that the mixin concept of flavors was such a valuable
tool for factoring object functionality that inheritance from multiple
supertypes was essential.  This idea of inheriting from several
mixins, each of which knows how to do something and encapsulates its
own state, led to the following inheritance rule: a new type inherits
all of the methods of its supertypes, but methods for the new type
cannot refer to instance variables from the supertypes (even though
those variables do exist in the new composite object.) This does not
cause problems, because when operations for the supertypes are passed
to the object, the methods which handle them {\it can} reference the
appropriate instance variables.  Since the names of instance variables
are never inherited, conflicts cannot occur between names in the
various supertypes.

This treatment of instance variable names was also motivated by our
decision to follow Scheme and make Oaklisp lexically scoped.  Oaklisp
not only benefits from the conceptual correctness which results from
being able to close methods at compile time, but takes full advantage
of tail-recursion and the lack of search associated with variable
references.  Once again, we decided to carry a principle to its
extreme, and say that {\it all} variable references must be resolved
at compile time, which results in both a simpler compiler and faster
execution of code.  Although this decision sounds intolerable for
users, it actually represents a shift of functionality from the
compiler to the error-handling system and user-interface, both of
which we decided to make unusually powerful in our Oaklisp
implementation.

Another principle which we borrowed from Scheme is anonymity.  The
lack of coupling between names and objects gives the system a degree
of modularity and flexibility that would otherwise be difficult to
achieve.  For example, if a type is redefined, old instances of that
type will still have pointers to the old type descriptor.  Operations
on both kinds of object will be handled by the correct methods, and
when the last instance of the old type goes away, the old type
descriptor will also be garbage collected.

The portion of Oaklisp that has been described so far can be
considered its kernel.  The portion that follows can mostly be
implemented as methods at the user level.  It is interesting to note
that the dynamic variables and mutable binding contours which are
described below can be built on top of a bare lexical Lisp kernel.
However, the Oaklisp kernel is not a usable system, since we
intentionally stripped it down knowing that the lost facilities would
be replaced at a higher level.

The first addition is a mutable binding contour facility based the
locale structures of T.\footnote{In place of locales we could have
implemented a simple top-level binding environment.} Oaklisp locales
are objects that accept messages which install and look up names.
Locales can have multiple superiors which are recursively searched if
a name can't be found.  Unlike T locales, Oaklisp locales are not
associated with textual binding contours that can interact with
\df{let}'s and generate ambiguities.  Moreover, a reference to an
undefined name creates an error, which means that forward references
to uninstalled names are impossible.

The second addition is a dynamic scoping facility that knows how to
deal with \df{catch} and \df{throw}.  The new dynamic variables are
entirely separate from the static variables, and are always textually
distinguishible from static variables to avoid confusion.  Dynamic
variables use deep-binding in our implementation so that they will
behave correctly when there is more than one process.  The
implementation of dynamic variables is an issue since we decided to
follow the Lisp Machine and implement light-weight processes that
share the same address space to expedite data sharing and fast context
switching.

Our final design decision was also influenced by the Lisp Machine.
Error handling in Oaklisp is designed to take maximum advantage of the
type inheritance mechanism that is built into the language.  A
complete hierarchy exists of all the types of system errors.  When an
error occurs, an instance of that error type is created, and a message
is sent to the error object asking for a handler to take control.  The
default message causes the debugger to be invoked.  However, each
process has some dynamic state which can be modified with the
\df{condition-bind} construct to cause a different handler to be
invoked when a particular error occurs at a particular time.  This
mechanism brings the full power of the language to bear on the problem
of resolving errors, and is the reason that we felt we could make the
language itself so strict with respect to variable references.  Since
our implementation of Oaklisp runs on the Lisp Machine and the
Macintosh, it was no problem to delegate authority for reporting and
resolving unbound variable references to the user-interface, which
uses menus and dialog boxes to determine the user's intentions.  If
the user sets a switch that indicates that he wants unbound names to
be automatically installed in the innermost locale, then the interface
merely creates an error-handler to perform that function, and the user
is not bothered again.
