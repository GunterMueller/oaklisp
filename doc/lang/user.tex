\chapter{User Interface} \label{user}

The Oaklisp user interface currently consists of a read-eval-print
loop and a simple debugging facility.

Errors land the user into a recursive evaluation loop in which special
restart handlers are available.  Our implementation includes
mechanisms for inspecting objects and tracing functions calls.


\section{The Top Level Environment}

All expressions must be evaluated with respect to a particular naming
environment.  The read-eval-print loop uses the locale specified by
the fluid variable \df{current-locale}.  The Oaklisp system boots up
with this variable bound to \df{user-locale}.  Other useful name
spaces are \df{scheme-locale}, \df{system-locale}, and
\df{compiler-locale}.

Several fluid variables are used to keep a short history of the
dialogue conducted by the top level evaluator.  The most useful of
these is \dffl{*}, which contains the value produced by the most
recent user expression.  The value of this variable is rolled back
into \dffl{**} and then into \dffl{***} to provide access to the three
most recent values.  Similarly, there are three copies of \dffl{+} and
\dffl{?} that provide access to recent expressions that were read in
and to their form after macro expansion.

The switch \dffl{fancy-references} controls the printing of anonymous
objects.  When this switch is turned off, an object usually prints out
something like this: \texttt{\#<op 806>}.  This format indicates the type
of the object, and provides a weak pointer that can be derefenced with
\df{object-unhash} to get the object.  When the
\dffl{fancy-references} switch is turned on, the printer attempts to
generate an expression that will evaluate to the object in the current
locale.  For example, the above operation might print out as \texttt{\#<op
(setter car) 806>}.  The default value for this switch is
\df{\#f}, but it is briefly switched on by \df{describe}.

Two more fluid variables that are frequently used at the top level are
\dffl{print-length} and \dffl{print-level}, which are normally set to
small integer values in order to abbreviate the printing of long
lists, but which can be set to \df{\#f} in order to enable exhaustive
printing.

%%
% 
% The read-eval-print loop uses the \df{eval} function to perform the
% actual evaluation of expressions.  The \df{eval} function in turn
% calls one of several functionally equivalent evaluation functions in
% our implementation: \df{interpreter-eval}, \df{compiler-eval}, and
% \df{hybrid-eval}.  The variable \dffl{top-level-evaluator}
% specifies which one should be used.  \df{hybrid-eval} is the default
% value in worlds that contain the compiler.  This evaluator minimizes
% top level latency by only using the compiler for expressions that
% contain \df{add-method} forms.
% 
% 


\section{Miscellaneous Functions}

There are some other very useful functions that are part of the user
interface.

\op{apropos}{word [place]}
\doc{Returns either variables or symbols containing \emph{word},
depending on \emph{place}, which can be a locale or \df{symbol-table}.
\emph{place} defaults to \dffl{current-locale}.}

\oop{\%gc}
\doc{Collect garbage.  This does not collect garbage in ``static
space,'' but it is exceedingly unlikely that there is any there.}
\oop{\%full-gc}
\doc{Collect more garbage.  This does collect garbage from ``static
space,'' but more importantly, it put everything not freed into static
space, so it need not be transported in future normal garbage
collections.}



\section{Debugging}


The following special forms can be used to trace the
execution of an operation.

\sform{trace-variable-in}{global-var}
\doc{Puts a trace on the operation stored in \emph{global-var}, causing
a message to be printed every time the operation is called.}

\sform{trace-variable-out}{global-var}
\doc{Puts a trace on the operation stored in \emph{global-var}, causing
a message to be printed every time a call to the operation returns.}

\sform{trace-variable-in-out}{global-var}
\sform{untrace-variable}{global-var}


Objects can be examined in detail with the \df{describe}
function, which prints the object and its type with
\dffl{fancy-references} turned on, followed by the object's
internal state.  The internal state is organized as instance-variable
blocks from the object's various component types.  An object's
internal state usually contains anonymous objects whose printed
representation includes weak pointers which can be dereferenced using
\df{object-unhash}.  Together, \df{describe} and \df{object-unhash}
constitute a simple but effective inspector.

To simplify this process \texttt{describe} applied to an integer which is
the \texttt{object-hash} of some object will describe that object.  In
other words, \texttt{describe} can be applied to the numeric ID in an
object's printed representation.

\op{describe}{object}
\doc{Prints out lots of stuff about \emph{object}.}

\op{object-unhash}{i}
\doc{Dereferences the weak pointer \emph{i}.}


When an error occurs in our implementation of Oaklisp, the
user is thrown into a recursive evaluation loop whose dynamic context
is nested inside that of the error.  Several restart handlers are
typically available in a recursive evaluation loop, and the \df{ret}
function is the mechanism for invoking one of these handlers.
\df{call/cc} can be used to preserve an error context when it might
be useful to restart the computation at a later time.


\op{ret}{n \dt args}
\doc{Invokes restart handler \emph{n}, as specified by the list of
handlers printed out by a subordinate evaluation loop.  \texttt{(ret 0)},
which returns control to the top level evaluation loop, is always in
effect.}

The following dialogue with Oaklisp illustrates some of these
points.

\begin{verbatim}
Oaklisp 1.0  - (C) 1987 Barak Pearlmutter and Kevin Lang.
Oaklisp evaluation loop.
  Active handlers:
  0: Return to top level.

> (with-open-file (inf "fone.nums" in) (car (read inf)))

Error: Error opening "fone.nums" for reading.
Oaklisp evaluation loop.
  Active handlers:
  0: Return to top level.
  1: Retry opening file (argument for different file name).
  2: Return to debugger level 1.

>> (call/cc identity)                ;get error context.

>> (set foo (fluid *))               ;stash it away.

>> (ret 0)                           ;back to top level.

   Invoking handler Return to top level..

> (describe foo)                    ;inspect continuation.

#<Op FOO 798> is of type #<Type OPERATION 801>.

 from #<Type 801>:
  LAMBDA? : #<Object 802>           ;what's this thing?
  CACHE-TYPE : 0
  CACHE-METHOD : 0
  CACHE-TYPE-OFFSET : 0

> (describe (object-unhash 802))

#<Object 802> is of type #<Type %METHOD 803>.

 from #<Type 803>:
  THE-CODE : #<VLmixin 804>
  THE-ENVIRONMENT : #<VLmixin 805>

> (foo 0)                    ;re-enter error context.

>> (ret 1 "phone.nums")      ;resume computation

   Invoking handler Retry opening the file ...

268-7598                     ;got that phone number!

> (exit)

Oaklisp stopped itself...
\end{verbatim}

Using the error system effectively is an important part of providing
the user with a helpful interface.  Details on the error system can be
found in section \ref{errors}.
