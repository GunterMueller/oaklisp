\chapter{Stack Discipline}

This chapter describes how the stacks are organized at the logical
level: how temporaries are allocated, how functions call and return
work, how escape objects (used in the implementation of catch and
throw) work, and how stack snapshots (used in the implementation of
call/cc) work.

\section{Stack Overview}

The Oaklisp bytecode machine has a two-stack architecture.  The 
\emph{value stack} contains arbitrary references and is used for storing
temporary variables, passing arguments, and returning results.  The
\emph{context stack} is used for saving non-variable context when
calling subroutines.  Only context frames are stored on the context
stack.  This two stack architecture makes tail recursion particularly
fast, and is in large part responsible for the speed of function call
in this implementation.

Most of the bytecodes are the usual sort of stack instructions, and
use only value stack, for instance \df{plus} and \texttt{(swap
$n$)}\index{\texttt{swap}}.  All arguments are passed on the value stack,
and the value stack is \emph{not} divided into frames.  Methods consume
their arguments, returning when they have replaced their arguments
with their result or tail recursing when they have replaced their
arguments with the appropriate arguments to the operations they are
tail recursing to.

Under the current language definition there is no multiple value
return, although the bytecode architecture admits such a construct.
There are facilities for variable numbers of arguments, which are
described in section \ref{sec:varargs}.


\newenvironment{stackphoto}{\begin{center}\begin{tabular}{|c|}
$\vdots$\\\hline}{\end{tabular}\end{center}}

\section{Method Invocation/Return}

When a method is to be invoked, the arguments and operation are
assembled on the value stack in right to left order, ie.\ the
rightmost argument is pushed first and the operation is pushed last.
Let us walk through the invokation of \texttt{(f x y z)}, where \texttt{f}
is on operations which is being passed three arguments.  Since we
evaluate right to left, first we push \texttt{z}, thus:
\begin{stackphoto}
\tt z \\\hline
\end{stackphoto}
continuing, we push the rest of the arguments and the operation, until
the stack is of this form.
\begin{stackphoto}
\tt z \\\hline
\tt y \\\hline
\tt x \\\hline
\tt f \\\hline
\end{stackphoto}
A \df{(store-nargs 3)} instruction is now executed to place the number
of arguments in the \df{nargs} register, and one of the \df{funcall}
instructions is executed, which variant depending on whether this is a
tail recursive call.  If this is not a tail recursive call, the
\df{funcall} instruction first pushes a frame containing the contents
of the \df{current\protect\_method}, \df{bp} and \df{env} registers
and a return \df{pc} onto the context stack.  The instruction then
examines the top two values, \texttt{f} and \texttt{x}, and looks \texttt{f} up
in the \df{operation-method-alist} of the type of \texttt{x}, potentially
also scanning the supertypes until it finds the appropriate method to
be invoked.  This method is placed in the \df{current\protect\_method}
register, the method's environment is placed in the \df{env} register,
the \df{pc} is set to the beginning of the method's code block, and
the address of the appropriate instance variable frame within \texttt{x} is
placed in the \df{bp} register.  The \df{funcall} instruction leaves
the value stack and \texttt{nargs} register unchanged:
\begin{stackphoto}
\tt z \\\hline
\tt y \\\hline
\tt x \\\hline
\tt f \\\hline
\end{stackphoto}

The first thing the code block of the resultant method executes is one
of the \df{check-nargs} instructions, in this case perhaps {\tt
(\df{check-nargs} 3)}.  A \texttt{(\df{check-nargs} $n$)} instruction
tests if \df{nargs} is $n$, trapping if not.  After that, it pops the
operation \texttt{f} off the stack.  By leaving the operation to be
popped off by the \df{check-nargs} instruction rather than the
\df{funcall} instruction, when an an incorrect number of arguments is
detected the operation is still available to the error system.  The
\df{return} instruction pops the top frame off the context stack,
loads the popped context into the processor, and continues execution.
Before a
\df{return} is executed all of the arguments have been consumed and
the result is the only thing left on the stack,
\begin{stackphoto}
 (f x y z) \\\hline
\end{stackphoto}

\section{The Context Stack}

The only things that can be stored on the context stack are context
frames, which each have four values, as below.  The \df{push-cxt}
instruction pushes a context frame onto the context stack.  It takes
an inline argument, which is the relative address of the desired
return point.  This allows a context to be pushed whenever convenient,
perhaps before the assembly of arguments begins.  Earlier in the
implementation process there was only one variant of the \df{funcall}
instruction, which was tail recursive.  Non tail recursive calls were
compiled as a \df{push-cxt} followed by a \df{funcall-tail}, but
because this sequence occured so frequently a combined instruction was
implemented to improve code density.

\begin{stackphoto}\hline
  \tt \df{pc}  \\ \hline
  \tt \df{bp}  \\ \hline
  \tt \df{env} \\\hline
  \tt \df{current\protect\_method} \\\hline\hline
  \tt pc  \\ \hline
  \tt bp  \\ \hline
  \tt env \\\hline
  \tt current\_method \\\hline\hline
  \tt pc  \\ \hline
  \tt bp  \\ \hline
  \tt env \\\hline
  \tt current\_method \\\hline
\end{stackphoto}

Actually, the \df{pc} stored in the context stack is not a raw pointer
to the next instruction but rather the offset from the beginning of
the current code block, stored as a fixnum.  This makes the
\df{return} instruction slightly slower, as the actual return pc must
be recomputed, but simplifies the garbage collector.  The \df{bp} is
analogously stored with a tag of \df{locative} so that the garbage
collector need not treat it specially.  This would cause a problem if
the current object were reclaimed and afterwards had one of its
instance variables refered to, as all that would be left of the object
would be the solitary cell that the saved bp was pointing to, and the
rest of the relevent instance variable frame would be gone.  This is
avoided by having the compiler ensure that a reference to the object
in question is retained long enough.
