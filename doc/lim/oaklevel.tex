% This file is part of Oaklisp.
%
% This program is free software; you can redistribute it and/or modify
% it under the terms of the GNU General Public License as published by
% the Free Software Foundation; either version 2 of the License, or
% (at your option) any later version.
%
% This program is distributed in the hope that it will be useful,
% but WITHOUT ANY WARRANTY; without even the implied warranty of
% MERCHANTABILITY or FITNESS FOR A PARTICULAR PURPOSE.  See the
% GNU General Public License for more details.
%
% The GNU GPL is available at http://www.gnu.org/licenses/gpl.html
% or from the Free Software Foundation, 59 Temple Place - Suite 330,
% Boston, MA 02111-1307, USA


\chapter{Oaklisp Level Implementation}

Once the core of the language is up, the rest of the language is
implemented using the language core.  Some of these new language
constructs require some support from the bytecode emulator along with
considerable Oaklisp level support.  These include such features as
\df{call/cc} and its simple cousin \df{catch}.  Others are implemented
entirely in the core language without the use of special purpose
bytecodes; in this latter class fall things like infinite precision
integers (so called \emph{bignums}), fluid variables, and the error
system.

In this chapter we describe the implementation of these constructs,
albeit sketchily.  For more details, the source code is publicly
available.  We do not describe the implementation of locales or other
extremely high level features; read the source for the details, which
are quite straightforward.


\section{Fluid Variables} \label{fluid-impl}

Our implementation of fluid variables uses deep binding.  A shallow
bound or hybrid technology would presumably speed fluid variable
reference considerably, but they are used rarely enough that we have
not bothered with such optimizations.  In addition, shallow binding
interacts poorly with multiprocessing.

\gv{fluid-binding-list}
\doc{Hold an association list which associates fluid variables to
their values.  The \df{bind} construct simply pushes variable/value
pairs onto this list before executing its body and pops them off
afterwards.}

It would be easy to implement fluid variables using the unwind
protection facilities, but instead the abnormal control constructs
(\df{native-catch} and \df{call/cc}) are careful to save and restore
\df{fluid-binding-list} properly.  This avoids the overhead of using
the wind facilities and makes sure that (ignoring \df{wind-protect})
\df{fluid-binding-list} is only manipulated once for every abnormal
exit, no matter how many \df{bind} constructs are exited and entered
along the way.

\lo{\%fluid}{symbol}
\doc{This looks \emph{symbol} up on \df{fluid-binding-list}.  If it
is not found an error is signaled.  In contrast, \texttt{(setter
\%fluid)} silently adds new fluid variables to the end of the
association list, thus creating new top level fluid bindings.}



\section{Unwind Protection} \label{sec:oakwind}

In the presence of \df{call/cc}, a simple \df{unwind-protect}
construct a.\ la.\ Common Lisp does not suffice.  Because control can
enter a dynamic context which has previously been exited, symmetry
requires that if we have forms that get executed automatically when a
context is abnormally exited, we must also have ones that get executed
automatically when a context is abnormally entered.  For this purpose
the system maintains some global variables that reflect the state of
the current dynamic context with respect to these automatic actions.

\gv{\%windings}
\doc{This is a list of wind/unwind action pairs, one of which is
pushed on each time we enter a \df{dynamic-wind} and poped off when we
leave it.  The wind/unwind action pairs are of the form
\texttt{(\emph{after} \emph{before} . \emph{saved-fluid-binding-list})}
where \emph{before} and \emph{after} are operations, guards to be
called when leaving and entering this dynamic context respectively,
and \emph{saved-fluid-binding-list} is the appropriate value for
\df{fluid-binding-list} when calling these guard operations.}

\gv{\%wind-count}
\doc{To reduce \df{find-join-point}'s complexity from quadratic to
linear, we maintain \df{\%wind-count} $=$ \texttt{(length \%windings)}.}


\section{Catch} \label{sec:oakcatch}

The format of catch tags is describe in section \ref{sec:ctagform}.
The simplest implementation of \df{native-catch} would have the
\df{native-catch} macro expand into something that executed the
appropriate unwind protect actions and restored the fluid binding list
before resuming execution.  Regretably, the unwind protect actions can
themselves potentially \df{throw}, so the stacks must not be chopped
off until after the unwind protect actions have been completed.  For
this reason the \df{throw} operation doesn't just call the \df{throw}
instruction, but first performs all the appropriate unwind protect
actions.  Along with stack heights, the catch tag contains
\df{saved-wind-count}, which is used to compute how many elements of
\df{\%windings} must be popped off and called, and
\df{saved-fluid-binding-list}, which is restored immediately before
the stacks are actually chopped off.


\section{Call/CC}

The \df{call/cc} construct is just like \df{native-catch}, except that
the saved stack state isn't just some offsets but is an entire stack
photo (see section \ref{sec:stackimpl}), and that not only unwinding
but also rewinding actions might need to be done.  Because the winding
actions might \df{throw}, it is necessary for the unwind actions to be
executed in the stack context where the continuation is invoked, and
similarly the rewind actions must be executed in the destination stack
context.

\gv{\%\%join-count}
\gv{\%\%new-windings}
\gv{\%\%new-wind-count}
\gv{\%\%cleanup-needed}
\doc{These global are used to pass information about which rewind
actions need to be executed by the destination of the continuation,
since the normal parameter passing mechanisms are not available.  This
would have to be done on a per processor basis in a multithreaded
implementation.}

Continuations contain \df{saved-windings} and \df{saved-wind-count}
instance variables, which have the values of \df{\%windings} and
\df{\%wind-count} at the time the \df{\%call/cc} was entered.  Before
the continuation is actually invoked and the destination stack photos
restored, the highest join point between current and the destination
winding lists is found, and all the unwind actions needed to get down
to the join point are executed.  Then the stack photo is restored, and
in the destination context the rewinding actions are done to get up
from the join point to the destination point.


\section{The Error System}

The error system is pretty complete, but is actually not only easy to
use, but also intuitive and fun, particularly at the user level.

\mc{error-return}{message \dt body}
\doc{Evaluates \emph{body} in a dynamic context in which a restart
handler is available that can force the form to return.  The handler
is identified by \emph{string} in the list of choices printed out by
the debugger.  If the handler is invoked by calling \df{ret} with an
argument in addition to the handler number, the \df{error-return} form
returns this additional value; otherwise it returns \df{\#f}.  If no
error occurs, an \df{error-return} form yields the value of 
\emph{body}.}

\mc{error-restart}{message let-clauses \dt body}
\doc{Acts like a \df{let}, binding the \emph{let-clauses} as you would
expect, except that if an error occurs while evaluating \emph{body},
the user is given the option of specifying new values for the
variables of the \emph{let-clauses} and starting \emph{body} again.
This is implemented with a \df{native-catch} and some tricky restart
handlers that get pushed onto \texttt{(fluid restart-handlers)}.}

\fv{restart-handlers}
\doc{A list of actions that the user can invoke from the debugger in
order to restart the computation at various places.  Not normally
manipulated by user code.}
\fv{debug-level}
\doc{The number of recursive debuggers we're inside.  Zero for the top
level.  Not normally manipulated by user code.}

\mc{catch-errors}
{\lpar error-type $[$error-lambda $[$non-error-lambda$]]$\rpar \dt body}
\doc{Evaluates \emph{body}.  If an error which is a subtype of 
\emph{error-type} occurs, \df{\#f} is returned, unless \emph{error-lambda} is
given, in which case it is called on the error object.  If no error
occurs then the result of evaluating \emph{body} is returned, unless
\emph{non-error-lambda} is provided in which case it is called on the
result of the evaluation of \emph{body} within the context of of the
error handler, and the resultant value returned.}

\mc{bind-error-handler}{\lpar error-type handler\rpar \dt body}
 \doc{This binds a handler to errors which are subtypes of 
\emph{error-type}.  When such an error occurs, an appropriate error object
is created and \emph{handler} is applied to it.}

\op{invoke-debugger}{error}
\doc{This error handler, when sent to an error object, invokes the debugger.}

\op{remember-context}{error after-op}
\doc{Used to make an error remember the context it occured in, so that
even after the context has been exited the error can still be
proceeded from, or the debugger can be entered back at the error
context.  This should always be called tail recursively from a
handler, and after it stashes away the continuation it calls 
\emph{after-op} on \emph{error}.  Of course, \emph{after-op} should never
return.}

\op{invoke-in-error-context}{error operation}
\doc{Go back to the context in which \emph{error} occured and invoke
\emph{operation} there.}

\op{report}{error stream}
\doc{Write a human readable account of the error to \emph{stream}.
Controlled studies have shown that error messages can never be too
verbose.}

\op{proceed}{error value}
\doc{Proceed from \emph{error}, returning \emph{value}.  Of course, it
is actually the call to \df{signal} that returns \emph{value}.}


\op{signal}{error-type \dt args}
\doc{This signals creates an error of type \emph{error-type} with
initialization arguments \emph{args}.  It then scans down \texttt{(fluid
error-handlers)} until it finds a type of error which is a supertype
of \emph{error-type}, at which point it sends the corresponding handler
to the newly minted error object.  If the handler returns, that value
is returned by the call to \df{signal}.  One day we'll add a way
for a handler to refuse to handle an error, in which case the search
for an applicable handler will proceed down the list.}

\fv{error-handlers}
\doc{An association list of mapping error types to error handlers.
Users should not touch this directly.}

Of course, there are a large number of types of errors used by the
system.  A few of the more useful to know about are:

\ty{general-error}
\doc{The supertype of all errors.  Abstract.}
\ty{proceedable-error}
\doc{The supertype of all errors that can be recovered from.  Abstract.}
\ty{fs-error}
\doc{File system error.  Abstract.  It has all kinds of subtypes for
all the different possible file system error conditions.}
\ty{error-opening}
\doc{Abstract.  Signaled when a file can't be opened for some reason.
Proceeding from this kind of error with a string lets you try opening
a different file.}
\ty{operation-not-found}
\doc{Signaled when an operation is sent to an object that can't handle
it.  Proceeding from this kind of error will return a value from the
failed call.}
\ty{nargs-error}
\doc{Signaled when there are an incorrect number of arguments passed
to a function.  Proceeding from this will return a value from the
failed call.  Abstract}
\ty{nargs-exact-error}
\doc{Signaled when there are an incorrect number of arguments passed
to a method that expects a particular number of arguments.}

\ty{nargs-gte-error}
\doc{Signaled when there are an insufficient number of arguments
passed to a method that can tolerate extra arguments.}

\ty{infinite-loop}
\doc{Signaled when an infinite loop is entered.  User programs may
wish to signal this as well.}

\ty{read-error}
\doc{Some kind of reader syntax error.  Abstract.  There are about
fifty million subtypes, corresponding to all the different constructs
that can be malformed, and all the different ways in which they can be
malformed.  We probably went a little overboard with these.}

\ty{user-interrupt}
\doc{Oaklisp received a DEL signal.  Through a convoluted series of
events in which the UNIX trap handler sets the variable
\df{\protect\_del\protect\_}, which is detected by the bytecode
emulator which pretends that a \df{noop} instruction failed and passes
the \df{nargs} register to the Oaklisp trap handler which salts the
old \df{nargs} away for restoration upon return and signals this error
type, the user usually lands in the debugger after typing Control-C.}




\section{Numbers}

Small integers (between $-2^{29}$ and $2^{29}-1$ inclusive) are
represented as immediates of type \df{fixnum} and handled directly by
microcode.  When arithmetic instructions trap out, due to either their
arguments not being \df{fixnum}s or to overflow, an Oaklisp operation
corresponding to the bytecode is called.  Most of these operations are
written in terms of other bytecodes, and should never be shadowed.
For instance,
\begin{verbatim}
(add-method (subtract/2 (number) x y)
  (+ x (- y)))
\end{verbatim}
defines subtraction in terms of negation and addition.  The trap code
also handles fixnum overflow, promoting the operands to \df{bignum}s
and dispatching appropriately.  The only really primitive operations,
which must handle all types of numbers, are \df{<}, \df{=},
 \df{minus}, \df{negative?}, \df{plus/2}, \df{times/2}, \df{/},
 \df{/r}, \df{quotient}, \df{remainder}, \df{quotientm} and
 \df{modulo}.  Whenever a new type of number is defined, methods for
all of the above operations should be added for it, unless the new
type is not a subtype of \df{real}, in which case methods wouldn't
make sense for \df{<}, \df{negative?}, and perhaps \df{quotient},
 \df{remainder}, \df{quotientm} and \df{modulo}.



\section{Vectors and Strings}

Rather than being built into the emulator, vectors are defined
entirely within Oaklisp, albeit with some rather low level constructs.

\ty{variable-length-mixin}
\doc{This type provides a variable amount of stuff at the end of its
instances.  When a type has this mixed in, whether immediately or deep
down in the inheritance tree, it always takes an extra initialization
argument which says has long the variable length block at the end
should be.  This is mixed into such system types as
\df{\%code-vector}, \df{stack-segment}, and \df{\%closed-environment}.

In general, \df{variable-length-mixin} is used at the implementation
level only and should never appear in user code.  Typically if you
think you want a subtype of \df{variable-length-mixin}, what you
really want is an instance variable bound to a vector.}

\lo{\%vref}{variable-length-object n}
\doc{This is the accessor operation to get at the extra cells of
subtypes of \df{variable-length-mixin}.  It is used in the
implementation of variable length structures, and in things like
\df{describe} that look at their internals.}

\ty{simple-vector}
\doc{This is a subtype of \df{vector} with \df{variable-length-mixin}
added and an appropriate \df{nth} method defined.}

\discuss{Characters are packed into strings more densely than one
character per reference, so strings are not just vectors with odd
print methods; they also have accessor methods which unpack characters
from their internals.  Unfortunately, it is not possible to pack four
eight bit characters into a single reference without violating the
memory format conventions by putting something other than
\framebox{\texttt{0 0}} in the tag field.  We could pack four seven bit
characters into each reference, but some computers use eight bit
fonts, and the characters within the string would not be aligned
compatibly with C strings.  We therefore use the following somewhat
wasteful format.}

\ty{string}
\doc{This is a subtype of \df{simple-vector} with the \df{nth} method
shadowed by one that packs three eight bit characters into the low 24
bits of each fixnum, in littleendian order.  The unused high bits of
each word are set to zero to simplify equality testing and hash key
computation.  No trailing null character is required, although one is
present two thirds of the time due to padding.  Below is the string
\texttt{"Oaklisp Rules!"} as represented in memory.}
\begin{center}
\begin{tabular}{|c|c|c|c|c|}\hline
31 \ldots 26 & 25 \ldots 18 & 17 \ldots 10 & 9 \ldots 2 & 1 0 \\\hline\hline
\multicolumn{5}{|c|}\emph{string} \\\hline
\multicolumn{4}{|c|}{\emph{object length:} 8} & 0 0 \\\hline
\multicolumn{4}{|c|}{\emph{string length:} 14} & 0 0 \\\hline
0 0 0 0 0 0&\tt\#$\backslash$k   &\tt\#$\backslash$a    &\tt\#$\backslash$O&0 0 \\\hline
0 0 0 0 0 0&\tt\#$\backslash$s   &\tt\#$\backslash$i    &\tt\#$\backslash$l&0 0 \\\hline
0 0 0 0 0 0&\tt\#$\backslash$R   &\tt\#$\backslash$space&\tt\#$\backslash$p&0 0 \\\hline
0 0 0 0 0 0&\tt\#$\backslash$e   &\tt\#$\backslash$l    &\tt\#$\backslash$u&0 0 \\\hline
0 0 0 0 0 0&\tt\#$\backslash$null&\tt\#$\backslash$!    &\tt\#$\backslash$s&0 0 \\\hline
\end{tabular}
\end{center}


\section{Symbols}

We do not use any of the fancy techniques used by older dialects, like
oblists or symbol buckets.  Instead, the standard hash table facility
is used for the symbol table.

\heady{symbol-table}{}{Generic Hash Table}
\doc{Maps strings to symbols, using \df{string-hash-key} to compute
the hash and \df{equal?} to compare strings for equality.}

\op{intern}{string}
\doc{Returns a symbol with print name \emph{string} by looking it up
in the \df{symbol-table} and making and installing a new symbol if it
isn't found.  Strings passed to \df{intern} should never be side
effected afterwards or the symbol table could be corrupted.}

\fv{print-escape}
\doc{This flags whether symbols with weird characters in them should
be with the weird characters escaped.  It also applies to strings.}

\fv{symbol-slashification-style}
\doc{This flag is only relevent if \texttt{(fluid print-escape)} is on.
With the value \df{t-compatible} then the empty symbol is printed as
\texttt{\#[symbol ""]} and all other symbols requiring escaping are
printed with a \texttt{$\backslash$} character preceding every character of the
symbol.  With any other value, escaped symbols are delimited by
\texttt{|} characters and internal characters \texttt{$\backslash$}
and \texttt{|} are
preceded by \texttt{$\backslash$}.}



\section{Variable Numbers of Arguments} \label{sec:varargs}

The formal parameter list of a method is permitted to be improper,
with the terminal atom being a magic token representing the rest of
the arguments.  The only legal use for this magic token is as the
terminal member of an improper argument list of a tail recursive call,
and as an argument to the special form \df{rest-length}.  Methods that
accept a variable number of arguments must exit tail recursively and
must pass along their magic token in their tail recursive call, unless
they know that they actually received no extra arguments.

\sform{rest-length}{varargs-token}
\doc{Returns the number of trailing arguments represented by
\emph{varargs-token}.}

For example, this is legal,
\begin{verbatim}
(define (okay x y . rest)
  (if (zero? (rest-length rest))
      'nanu-nanu
      (list 'you x y 'sucker . rest)))
\end{verbatim}
while the following are not, the first because it has an exit when
there might be extra arguments which does not pass the extra arguments
along tail recursively, and the second because it tries to pass along
the extra arguments in a non tail recursive position.
\begin{verbatim}
(define (not-okay x y . rest)
  (if (eq? x y)
      'nanu-nanu
      (list 'you x y 'sucker . rest)))

(define (also-bad x y . rest)
  (append (list 'you x 'sucker . rest) y))
\end{verbatim}

The implementation behind this is very simple: extra arguments are
ignored by the compiler, except that it emits a \df{check-nargs-gte}
in place of a \df{chech-nargs} at the top of the method code body and
does a little computation to figure out what the value to put in the
\df{nargs} register when it sees rest argument at the tail of a call.
When all the user wishes to do is pass the extra arguments along in
the way that the \df{make} method passes extra args along to
\df{initialize}, this mechanism is both convenient and efficient.
Sometimes the user needs to actually get into the extra arguments
though, so some operations are provided to make handling variable
numbers of arguments easier.

\op{consume-args}{value \dt extra}
\doc{Returns \emph{value}.}

\op{listify-args}{operation \dt args}
\doc{Calls \emph{operation} with a single argument, a list of 
\emph{args}.}

There is also a macro package that implements optional and keyword
arguments using these facilities, and the Scheme compatiblity package
redefines \df{add-method} so that, as required by the Scheme standard
\citep{R3RS}, extra arguments are made into a list.
