% This file is part of Oaklisp.
%
% This program is free software; you can redistribute it and/or modify
% it under the terms of the GNU General Public License as published by
% the Free Software Foundation; either version 2 of the License, or
% (at your option) any later version.
%
% This program is distributed in the hope that it will be useful,
% but WITHOUT ANY WARRANTY; without even the implied warranty of
% MERCHANTABILITY or FITNESS FOR A PARTICULAR PURPOSE.  See the
% GNU General Public License for more details.
%
% The GNU GPL is available at http://www.gnu.org/licenses/gpl.html
% or from the Free Software Foundation, 59 Temple Place - Suite 330,
% Boston, MA 02111-1307, USA


\chapter{Stack Machine Architecture}


\section{Registers in the Emulator}

This section describes the registers that make up the state of the
bytecode emulator, called the processor below.

\begin{description}

\item[\tt pc:] The program counter points to a half reference
address, and can not be accessed by register instructions.

\item[\tt val\_stk:] The top of the value stack.  Can not be accessed by
register instructions.

\item[\tt cxt\_stk:] The top of the context stack.  Can not be accessed by
register instructions.

\item[\tt bp:] The base pointer points to the base of the instance variable frame
of the current object.

\item[\tt env:] The current environment object is indexed into to find
locatives to lexically closed variables.

\item[\tt current\_method:] The method whose code is currently being
executed.  This is maintained solely to simplify garbage collection
and debugging.

\item[\tt nargs:] The number of args register is set before a function
call and checked as the first action within each function.

\item[\tt t:] Holds the cannonical truth object, \texttt{\#t}.

\item[\tt nil:] Holds the cannonical false object, \texttt{\#f}, which is also
used as the empty list, \texttt{()}.

\item[\tt fixnum\_type:] Holds the type of objects with a tag of fixnum.

\item[\tt loc\_type:] Holds the type of objects with a tag of locative.

\item[\tt subtype\_table:] Holds a table of the types of all the immediate
subtypes.  Currently only the first entry is used.

\item[\tt cons\_type:] Holds the \emph{cons-pair} type, the type of simple
conses which are directly manipulated by the processor.

\item[\tt env\_type:] Holds that type of environment vectors, used when
making new environment objects.

\item[\tt object\_type:] Holds the type \emph{object} which is at the root
of the type hierarchy.  Used when calling an operation with no
parameters.  This should not be necessary in the next version.

\item[\tt segment\_type:] Holds the type of stack segments, for use when
the stack is being copied into the heap.

\item[\tt argless\_tag\_trap\_table:] Holds a table of operations to be
called when various instructions fail.

\item[\tt arged\_tag\_trap\_table:] Holds a table of operations to be
called when various instructions fail.

\item[\tt boot\_code:] Holds the method to be called first thing at boot time.

\item[\tt uninitialized:] Holds the value that gets stuck into newly
allocated storage.

\item[\tt free\_point:] Holds the point at which the next heap object will
be allocated.  Not accessed directly by even the most internal Oaklisp
code, as the processor takes care of initialization and gc itself.

\item[\tt new.end:] Holds the point at which we've run out of storage.  An
attempt to allocate past here necessitates a garbage collection.  Not
directly accessed by even the most internal Oaklisp code.

\item[\tt next\_newspace\_size:] Holds the size in references of the
next new space to be allocated by the garbage collector.  This is
dynamically adjusted by the garbage collector, so there is usually no
need for it to be modified from the Oaklisp level.

\end{description}




\section{Instruction Set}
The instructions follow the same argument order conventions as the
language itself.  For example, \texttt{(\df{store-loc} \emph{loc ref})}
expects to get \emph{loc} on the top of the value stack and \emph{ref}
below it.  The instruction format
\begin{center} \begin{tabular}{|c|c|c|}\hline
 8 bits          & 6 bits & 2 bits \\\hline
 inline argument & opcode & 0 0    \\\hline
\end{tabular} \end{center}
leaves eight bits for an inline argument.  Instructions that do not
require any inline argument actually have ``argless instruction'' in
their instruction field and use the argument field to code for the
actual instruction.

Some instructions, eg.\ \df{load-imm}, require a complete arbitrary
reference as an inline argument.  This in incorporated, aligned,
directly in the instruction stream.  See section \ref{sec:codeblock}
for details.  Other instructions, in particular the long branches,
require more than an eight bit inline argument but do not need an
entire reference.  These instructions get a 14 bit inline argument by
using the space where the next instruction would normally go, with the
last two bits set to zero in case the argument ends up in the low half
of a word.

\newenvironment{itable}[1]{
\begin{center} \par \nopagebreak #1 \par \nopagebreak
\begin{tabular}{|l|c|l|l|l|} \hline
\emph{instruction} & \emph{inline arg} & \emph{initial stack}
 & \emph{final stack} & \emph{extra cell args}
\\\hline\hline}{\hline\end{tabular}\end{center}}

\newcommand{\icomment}[1]{\multicolumn{5}{|l|}{\parbox{4in}{#1}}\\\hline}

\begin{itable}{Arithmetic}
\df{plus}	& 		& 2 (fix,fix)	& 1 (fix)	& \\ \hline
\df{minus}	&		& 1 (fix)	& 1 (fix)	& \\ \hline
\df{subtract}	& 		& 2 (fix,fix)	& 1 (fix)	& \\ \hline
\df{times}	& 		& 2 (fix,fix)	& 1 (fix)	& \\ \hline
\df{mod}	& 		& 2 (fix,fix)	& 1 (fix)	& \\ \hline
\df{div}	& 		& 2 (fix,fix)	& 1 (fix)	& \\ \hline
\df{log-op}	& n (4 bits)	& 2 (fix,fix)	& 1 (fix)	& \\ \hline
\df{bit-not}	& 		& 1 (fix)	& 1 (fix)	& \\ \hline
\df{rot}	& 		& 2 (fix,fix)	& 1 (fix)	& \\ \hline
\df{ash}	& 		& 2 (fix,fix)	& 1 (fix)	& \\ \hline
\end{itable}

\begin{itable}{Predicates}
\df{eq?}& 		& 2 (ref,ref)	& 1 (bool)	& \\ \hline
\df{not}& 		& 1 (ref)	& 1 (bool)	& \\ \hline
\df{<0?}& 		& 1 (fix)	& 1 (bool)	& \\ \hline
\df{=0?}& 		& 1 (fix)	& 1 (bool)	& \\ \hline
\df{=}	& 		& 2 (fix,fix)	& 1 (bool)	& \\ \hline
\df{<}	& 		& 2 (fix,fix)	& 1 (bool)	& \\ \hline
\end{itable}

\begin{itable}{Control}
\df{branch}		& rel-addr	&		&	& \\ \hline
\df{branch-nil}		& rel-addr	& 1 (ref)	&	& \\ \hline
\df{branch-t}		& rel-addr	& 1 (ref)	&	& \\ \hline
\df{long-branch}	& 		&		&	& 0.5\\ \hline
\df{long-branch-nil}	& rel-addr	& 1 (ref)	&	& 0.5\\ \hline
\df{long-branch-t}	& rel-addr	& 1 (ref)	&	& 0.5\\ \hline
\df{return}		& 		&		&	& \\ \hline
\end{itable}

\begin{itable}{\df{catch} and \df{call/cc} Related}
\df{filltag}	&	& 1 (tag)	& 1 (tag)	& \\ \hline
\df{throw}	&	& 2 (tag,ref)	& 1 (ref)	& \\ \hline
\df{fill-continuation}&	& 1 (photo)	& 1 (photo)	& \\ \hline
\df{continue}	& 	& 2 (photo,ref)	& 1 (ref)	& \\ \hline
\end{itable}

\begin{itable}{Stack Manipulation}
\icomment{All stack references are zero-based.  \texttt{(swap 0)} is a
noop.  \texttt{(blast $n$)} \meq \texttt{(store-stack $n$)(pop 1)}.}
\df{pop}	& n	& n (refs)	&		& \\ \hline
\df{swap}	& n	& n (refs)	& n (refs)	& \\ \hline
\df{blast}	& n	& n (refs)	& n-1 (refs)	& \\ \hline
\df{blt-stack}& n,m	& n+m (refs)	& n (refs)	& \\ \hline
\icomment{8 bit ref splits to 4-bit n and m, which are $1 \ldots 16$.}
\end{itable}

\begin{itable}{Register Manipulation}
\df{store-reg} & register	& 1 (ref)	& 1 (ref)	& \\ \hline
\df{load-reg} & register	&		& 1 (ref)	& \\ \hline
\end{itable}

\begin{itable}{Addressing Modes}
\df{store-env} & offset	& 1 (ref)	& 1 (ref)	& \\ \hline
\df{store-stack} & offset	& 1 (ref)	& 1 (ref)	& \\ \hline
\df{store-bp} & offset	& 1 (ref)	& 1 (ref)	& \\ \hline
\df{store-bp-i} &  		& 2 (fix,ref)	& 1 (ref)	& \\ \hline
\df{contents} & 		& 1 (loc)	& 1 (ref)	& \\ \hline
\df{set-contents} &	 	& 2 (loc,ref)	& 1 (ref)	& \\ \hline
\icomment{The next two instructions are the same.}
\df{load-glo}& 		&		& 1 (ref)	& 1 (ref)\\ \hline
\df{load-imm}& 		&		& 1 (ref)	& 1 (ref)\\ \hline
\df{load-imm-fix} & n	&		& 1 (fix)	& \\ \hline
\df{load-env} & offset	&		& 1 (ref)	& \\ \hline
\df{load-stack} & offset	&		& 1 (ref)	& \\ \hline
\df{load-bp}	& offset	&		& 1 (ref)	& \\ \hline
\df{load-bp-i} & 		& 1 (fix)	& 1 (ref)	& \\ \hline
\icomment{Make a locative to the location $offset$ in beyond the
 \df{bp} register:}
\df{make-bp-loc} & offset	&		& 1 (loc)	& \\ \hline
\df{locate-bp-i} & 		& 1 (fix)	& 1 (loc)	& \\ \hline
\end{itable}

\begin{itable}{Memory Model and Tag Cleaving}
\df{get-tag}	& 		& 1 (ref)	& 1 (fix)	& \\ \hline
\df{get-data} & 		& 1 (ref)	& 1 (fix)	& \\ \hline
\df{crunch}	& 		& 2 (fix,fix:tag)& 1 (ref)	& \\ \hline
\df{load-type} & 		& 1 (ref)	& 1 (ref:type)	& \\ \hline
\df{load-length} & 		& 1 (ref)	& 1 (fix)	& \\ \hline
\icomment{The next two instructions are not currently used.}
\df{peek}	& 		& 1 (fix)	& 1 (fix:16-bit) & \\ \hline
\df{poke}	& 		& 2 (fix,fix:16-bit)&1 (fix:16-bit)&\\ \hline
\end{itable}

\begin{itable}{Misc}
\df{check-nargs} & n		& 1 (op)	&		& \\ \hline
\df{check-nargs-gte} & n	& 1 (op)	&		& \\ \hline
\df{store-nargs} & n		&		&		& \\ \hline
\df{noop}	& 		&		&		& \\ \hline
\df{allocate}  & 		& 2 (typ,len)	& 1 (ref)	& \\ \hline
\df{vlen-allocate}&	 	& 2 (typ,len)	& 1 (ref)	& \\ \hline
\df{funcall-tail} &	 	& 2 (op,obj)	& 1 (op,obj)	& \\ \hline
\df{funcall-cxt-br}& rel-addr& 2 (op,obj)	& 1 (op,obj)	& \\ \hline
\df{push-cxt}	& rel-addr	&		&		& \\ \hline
\df{push-cxt-long} &	 	&		&		& 0.5\\ \hline
\df{big-endian?} &	 	&		& 1 (bool)	& \\ \hline
\df{object-hash}&	 	& 1 (ref)	& 1 (fix)	& \\ \hline
\df{object-unhash}&	 	& 1 (fix)	& 1 (ref)	& \\ \hline
\df{gc}		& 		&		& 1 (ref)	& \\ \hline
\df{full-gc}	&		&		& 1 (ref)	& \\ \hline
\df{inc-loc}	&		& 2 (loc,fix)	& 1 (loc)	& \\ \hline
\end{itable}

\begin{itable}{List related instructions}
\df{cons}  & 		& 2 (ref,ref)	& 1 (ref)	& \\ \hline
\df{reverse-cons}  & 	& 2 (ref,ref)	& 1 (ref)	& \\ \hline
\df{car}  & 			& 1 (pair) 	& 1 (ref)	& \\ \hline
\df{cdr}  &	 		& 1 (pair)	& 1 (ref)	& \\ \hline
\df{set-car}  & 		& 2 (pair,ref)	& 1 (ref)	& \\ \hline
\df{set-cdr}  & 		& 2 (pair,ref)	& 1 (ref)	& \\ \hline
\df{locate-car}  & 		& 1 (pair) 	& 1 (loc)	& \\ \hline
\df{locate-cdr}  & 		& 1 (pair) 	& 1 (loc)	& \\ \hline
\df{assq}	& 		& 2 (ref,alist)& 1 (ref:pair/nil)& \\ \hline
\end{itable}


\section{Weak Pointers} \label{sec:weak}

Weak pointers allow users to maintain tenuous references to objects,
in the following sense.  Let $\alpha$ be a weak pointer to the object
\emph{foo}, found by executing the code \texttt{(object-hash \emph{foo})}.
This $\alpha$ can be dereferenced to yield a normal reference,
\evto{(object-unhash $\alpha$)}{\emph{foo}}.  However, if there is no
other way to get a reference to \emph{foo} then the system is free to
invalidate $\alpha$, so \evto{(object-unhash $\alpha$)}{\#f}.  In
practice, when the garbage collector sees that there are no references
to \emph{foo} except for weak pointers it reclaims \emph{foo} and
invalidates any weak pointers to it.

Weak pointers are implemented directly by bytecodes because the
emulator handles all details of storage allocation and reclamation
directly.  Weak pointers are represented by integers.  Each time call
call the \df{object-hash} is made the argument is looked up in the
\emph{weak pointer hash table}.  If no entry is found, a counter is
incremented and the value of that counter is returned.  An entry is
made in the \emph{weak pointer table} at an index corresponding to the
current value of the counter, so that the weak pointer can be used to
get back the original reference, and an entry is make in the weak
pointer hash table to ensure that if the weak pointer to the same
object is requested twice, the same number will be returned both
times.  After a garbage collection the weak pointer table is scanned
and entries to objects which have been reclaimed are discarded, the
weak pointer hash table is cleared, and the data in the weak pointer
table is entered into the weak pointer hash table.  Although these
algorithms are poor if objects with weak pointers to them are
frequently reclaimed, in practice this has not been a problem.
