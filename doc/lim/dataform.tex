% This file is part of Oaklisp.
%
% This program is free software; you can redistribute it and/or modify
% it under the terms of the GNU General Public License as published by
% the Free Software Foundation; either version 2 of the License, or
% (at your option) any later version.
%
% This program is distributed in the hope that it will be useful,
% but WITHOUT ANY WARRANTY; without even the implied warranty of
% MERCHANTABILITY or FITNESS FOR A PARTICULAR PURPOSE.  See the
% GNU General Public License for more details.
%
% The GNU GPL is available at http://www.gnu.org/licenses/gpl.html
% or from the Free Software Foundation, 59 Temple Place - Suite 330,
% Boston, MA 02111-1307, USA


\chapter{Internal Data Format}

This chapter describes how memory and tags are set up, and how this
implements the object semantics of the language.

\section{Tag Types} \label{immtags}

In an effort to reduce the complexity of the bytecode interpreter and
to simplify the system in general, there are only four tag types.
Tags are stored in the two low order bits of each reference thus
simplifying tag manipulation, particularly in the presence of indexed
addressing modes.

\begin{center}
\begin{tabular}{|c|c|c|l}
\cline{1-3}
\em 31 30 29 28 27 26 \ldots 11 10 9 8 & \em 7 6 5 4 3 2 & \em 1 0 &
 \multicolumn{1}{l}\emph{type} \\ \cline{1-3}
\multicolumn{2}{|c|}{twos complement integer} & 0 0 & fixnum \\ \cline{1-3}
data	&	subtype		& 1 0 & other immediate type \\ \cline{1-3}
\multicolumn{2}{|c|}{address}	& 0 1 & locative (pointer to cell) \\ \cline{1-3}
\multicolumn{2}{|c|}{address}	& 1 1 & reference to boxed object \\ \cline{1-3}
\end{tabular}
\end{center}

This tagging scheme, along with our object format, does not allow for
\emph{arbitrarily} scannable heaps (in which the divisions between
objects can be figured out starting the scan at any point in the
heap.)  In fact, if solitary cells are permitted, as they are in our
implementation, scanning the heap starting at the begining is not even
possible.  However, our garbage collector never needs to scan the heap
in such a fashion.  Note that there is no extra ``gc'' bit in every
word, but again, our garbage collector requires no such bit.


\section{Other Immediate Types}

References with a tag of  \framebox{1 0} use the next six bits to
specify a subtype.
\begin{center}
\begin{tabular}{|c|c|c|c|c|l}
\cline{1-5}
\emph{31 \ldots 24} & \emph{23 \ldots 16} & \emph{15 \ldots 8} &
\emph{7 \ldots 2} & \emph{1 0} & \multicolumn{1}{l}\emph{type} \\ \cline{1-5}
 reserved & font & ascii code & 0 0 0 0 0 0 & 1 0 & character\\ \cline{1-5}
\end{tabular}
\end{center}
Character is currently the only ``other immediate type.''  More may be
added later, in particular Macintosh handles.  (At one time weak
pointers were represented as their own immediate type, but they are
now represented using integers for compatibility with the Scheme
standard \cite{R3RS}.)


\section{Memory Structure}

Memory is a linear array of \emph{cells}, 32-bit aligned words.  These
cells are divided into two contiguous chunks: free cells and allocated
cells.  The \emph{free pointer} points to the division between these
two chunks, and it is incremented as memory is allocated.  When
allocating an object would push the free pointer beyond the limits of
memory, a garbage collection is performed.

The allocated portion of memory is divided into \emph{aggregate
objects} and \emph{solitary cells}.  Each aggregate object is a
contiguous chunk of cells.  The first cell of an object is a reference
to its type; if the type is \emph{variable length}, the second is
stores the length of the object, including the first two cells.  The
remainder of the cells of hold the instance variables.  Solitary cells
are cells that are not part of any object, but are the targets of
locatives.  Solitary cells are used heavily in the implementation of
variables.

A reference to an object consists of a pointer to that object with a
tag of \emph{boxed-object}.  References to solitary cells are locatives.
Furthermore, locatives may reference cells that are the instance
variables of objects.  If such an object is ever deallocated by the
garbage collector, all of the cells making up the object are made free
\emph{except} for those cells that are referenced by locatives, which are
not deallocated.  These become solitary cells.

\section{Representation of Specific Types}

Consider an object of type \emph{foo}, which is based on \emph{bar} and
\emph{baz}.  \emph{Bar} had instance variables \texttt{bar-1} and \texttt{bar-2}, baz
has instance variables \texttt{baz-1}, \texttt{baz-2} and \texttt{baz-3}, and \emph{foo}
has instance variable \texttt{foo-1}.  \emph{Foo} inherits the instance
variables of the types it is based on, but methods defined for type
\emph{foo} can not refer to these inherited variables.

Each type's local instance variables are stored contiguously, and in
order of lexical definition, in instances of that type, and of types
that inherit it; this allows variable reference to instance variables
to be resolved into offsets from the start of the relevent instance
variable frame at compile time.  Here is an instance of \emph{foo} as
it might actually be stored in memory:

\begin{center}
\begin{tabular}{|c|} \hline
 reference to type \emph{foo} \\ \hline\hline
 value of \texttt{foo-1} \\ \hline\hline
 value of \texttt{baz-1} \\ \hline
 value of \texttt{baz-2} \\ \hline
 value of \texttt{baz-3} \\ \hline\hline
 value of \texttt{bar-1} \\ \hline
 value of \texttt{bar-2} \\ \hline
\end{tabular}
\end{center}

Observe that instances of type \emph{foo} are divided into contiguous
chunks of instance variables, each inherited from a different
supertype.  When a type inherits another type through two different
routes, it still only inherits the instance variables
once.\footnote{This aspect of the language is in flux, and should not
be relied upon by users.} Furthermore, if the instance variables of
two types inherited by a third have the same names they are still
distinct instance variables.\footnote{This is in marked contrast to
ZetaLisp flavors--that's why variable references in flavors go through
mapping tables, resulting in considerable overhead.  There are also
important modularity considerations in favor of our scheme which are
beyond the scope of this document, but are discussed in detail in
\cite{SNYDER86}.} These semantics allow us to reference instance
variables very quickly, once the local instance variable block has
been located.  It also allows us to use the same compiled code for a
single method regardless of whether it is being invoked upon an
instance of the type it was added to or on an instance of an
inheriting type.


\section{System Types}

This section describes the format of various objects that are directly
referenced by the microcode,\footnote{Our microcode is C.} such as
code vectors and catch tags.

It should be emphasized that these system objects are full-fledged objects.
They have types which can be inherited and have their methods overridden,
just like any other object.  The only ``magic'' thing about these types is
that their local instance variables (ie. the system ones) must live at the
top of their memory representation, even when inherited.  This allows the
microcode to locate the values it needs without going through the type
heirarchy.

The only constraint this places on the user is that a type may not inherit
two types both of which are \emph{top-wired}, for obvious reasons.  For
example, it is impossible to make a type whose instances are both
operations and types.

\subsection{Methods}

A method has two instance variables which hold the code object
containing the code that implements the method and the environment
vector that holds references to variables that were closed
over.\footnote{Well, not all closed over variables.  Only ones above
the locale level.  Locale variable references are implemented as
inline references to value cells.}

\subsection{Environment Vectors}

Environment vectors have a block of cells, each of which contains a
locative to a cell.  When the running code needs to reference a
closed-over variable, it finds the location of the cell by indexing into
the environment vector.  This index is calculated at compile time, and such
references consume only one instruction.

Just as it is possible for a number of methods to share the same code,
differing only in the associated environment, it is also possible for
a number of methods to share the same environment, differing only in
the associated code.  Currently the compiler does not generate such
sophisticated constructs.

\subsection{Code Vectors} \label{sec:codeblock}

Code lives in vectors of integers, which are interpreted as
instructions by the bytecode emulator.  This format allows code to be
stored in the same space as all other objects, and allows the garbage
collector to be ignorant of its existance, treating code vectors like
any other vector.  Bytecodes are 16 bits long, with the low 2 bits
always 0.  Here is an example of some stuff taken from the middle of a
code vector.

\begin{center}
\begin{tabular}{|c|c|c|c|c|c|}
\multicolumn{6}{|c|}{$\vdots$}\\\hline
8 bit inline arg & 6 bit opcode & 0 0 &
8 bit inline arg & 6 bit opcode & 0 0 \\\hline
\multicolumn{2}{|c|}{14 bit instruction} & 0 0 &
8 bit inline arg & 6 bit opcode & 0 0 \\\hline
\multicolumn{2}{|c|}{14 bit relative address} & 0 0 &
8 bit inline arg & 6 bit opcode & 0 0 \\\hline
8 bit inline arg & 6 bit opcode & 0 0 &
8 bit inline arg & 6 bit opcode & 0 0 \\\hline
\multicolumn{2}{|c|}{14 bit instruction} & 0 0 &
\multicolumn{2}{c|}{14 bit instruction} & 0 0 \\\hline
\multicolumn{6}{|c|}{arbitrary reference used by last instruction of
 previous word} \\\hline
\multicolumn{2}{|c|}{14 bit instruction} & 0 0 &
8 bit inline arg & 6 bit opcode & 0 0 \\\hline
\multicolumn{6}{|c|}{$\vdots$}
\end{tabular}
\end{center}

Note the arbitrary reference right in the middle of code.  To allow
the garbage collector to properly handle code vectors, as well as to
allow the processor to fetch the cell efficiently, this reference must
be cell aligned.  When the processor encounters an instruction that
requires such an inline argument, if the pc is not currently pointing
to an aligned location then the pc is suitably incremented.  This
means that the assembler must sometimes emit a padding instuction,
which will be ignored, between instructions that require inline
arguments and their arguments.

An alternative that was used earlier in the design process was to
mandate that all instructions requiring inline arguments occur in a
position where the following reference can be fetched without
realigning the pc.  This requires sometimes inserting a padding
\texttt{noop} before an instruction that requires an inline argument,
and
analysis showed that the time required to process a \df{noop}
instruction is much greater than the time required to check if the low
bit of a register is on and increment that register if so.


\subsection{Endianity}

The logical order of the instructions in a code vector depends on the
endianity of the CPU running the emulator.  If the machine is big
endian, ie.\ addresses start at the most significant and of a word and
go down (eg.\ a 68000 or an IBM 370) then instructions are executed
left to right in the picture above.  Conversely, on a littleendian
machine (eg.\ a VAX) instructions are executed right to left.  Of
course, the Oaklisp loader has to be able to pack instructions into
words in the appropriate order.  The format of cold world loads is
insensitive to endianity, but binary world loads are sensitive to it,
so binary worlds are distributed in both big endian (with extensions
beginning with \df{.ol}) and little endian (with extensions beginning
with \df{.lo}) versions.

\oop{\%big-endian?}
\doc{This returns the endianity of the machine that Oaklisp is running
on.  Endianity is determined by the order in which instructions are
fetched, in other words, the order of two 16-bit words within a 32-bit
word.  This returns true if the first instruction fetched is from the
more significant half.}

\subsection{Stack Implementation} \label{sec:stackimpl}

Although the value and context stacks are logically contiguous, they
are sometimes physically discontinuous.  The instructions all assume
that stacks live in a designated chunk of memory called the stack
buffer.  They check if they are about to overflow or enderflow the
stack buffer, and if so they take appropriate actions to fill or flush
it, as appropriate, before proceeding.

If the stack buffer is about to overflow, most of it is copied to a
\emph{stack segment} which is allocated on the heap.  These overflown
segments form a linked list, so upon stack underflow the top segment
is removed from this list and copied back to the stack buffer.

There is one more circumstance in which the stack buffer is flushed.
The \df{call/cc} construct of Scheme \cite{R3RS} is implemented in
terms of \emph{stack photos,} which are snapshots of the current state
of the two stacks, and which can be restored in the future.  A
\df{fill-continuation} instruction forces the stack buffers to be
flushed and then copies references to the linked lists of overflow
segments into a continuation object.

Actually, in the above treatment we have oversimplified the concept of
flushing a stack buffer.  The emulator constant
\df{MAX\protect\_SEGMENT\protect\_SIZE} determines the maximum size of
any flushed stack segment.  When flushing the stack, if the buffer has
more than that number of references then it is flushed into a number
of segments.  This provides some hysteresis, speeding \df{call/cc} by
taking advantage of coherence in its usage patterns.  A possibility
opened by our stack buffer scheme, which we do not currently exploit,
is that of using virtual memory faults to detect stack buffer
overflows, thus eliminating the overhead of explicitly checking for
stack overflow and underflow.

As a historical note, an early version did not use a stack buffer but
instead implemented stacks as linked lists of segments which always
lived in the heap.  When pushing over the top of a segment, a couple
references were copied from the top of that segment onto a newly
allocated segment, providing sufficient hysteresis to prevent repeated
pushing and poping along a segment boundary from incurring inordinate
overhead.  Regretably, substantial storage is wasted by the hysteresis
and the overflow and underflow limits vary dynamically wereas in the
new system these limits are C link-time constants.  Presumably due to
these factors, in spite of its old world charm, timing experiments
between the old system and the new system were definitive.


\subsection{Escape Objects} \label{sec:ctagform}

In our implementation of Oaklisp we provide two different escape
facilities: \df{call/cc} and \df{catch}.  The \df{call/cc} construct
is that described in the Scheme standard \cite{R3RS}.  The \df{catch}
facility provides with user with a second class \emph{catch tag}, which
is valid only within the dynamic extent of the \df{catch}.

The implementation of catch tags is very simple: they contain heights
for the value and context stacks.  When a catch tag is thrown to, the
value and context stacks are chopped off to the appropriate heights.
The slot \df{saved-wind-count} is used for unwind protection and
\df{saved-fluid-binding-list} is used for fluid variables.  Details
are given in sections \ref{sec:oakcatch} and \ref{sec:oakwind}.

\begin{center}
\begin{tabular}{|c|}\hline
\emph{type:} escape-object \\\hline
\emph{value stack height:} 25 \\\hline
\emph{context stack height:} 19 \\\hline
\emph{saved wind count:} 3 \\\hline
\emph{saved fluid binding list:} \tt ((print-length . \#f) \ldots)\\\hline
\end{tabular}
\end{center}

Actually, there are two variants of \df{catch}.  In the regular
variant, which is compatible with T, the escape object is invoked by
calling it like a procedure, as in \texttt{(catch a (+ (a 'done) 12))}.
In the other variant, the escape object is not called but rather
thrown to using the \df{throw} operation, as in \texttt{(native-catch a
(+ (throw a 'done) 12))}.  Although the latter construct is slightly
faster, the real motivation for its inclusion is to remind the user
that the the escape object being thrown to is not first class.  In
order to ensure that an escape object is not used outside of the
extent of its dynamic validity, references to them should not be
retained beyond the appropriate dynamic context.

\subsection{Types}

Type objects are used when tracing up the type heirarchy in order to
find appropriate methods and bp offsets.  Since the types are used to
find methods, they must be system objects so that reference to their
instance variables can be done without sending them explicit messages.
The \df{operation-method-alist} maps from operations to methods
handled by the type itself, not any supertype.  The \df{type-bp-alist}
maps from types to offsets which are where the appropriate frame of
instance variables may be found.  The microengine uses a simple
move-to-front heuristic in an attempt to reduce the overhead of
searching these alists.  The \df{supertype-list} contains a list of
the immediate supertypes.  Supertypes by inheritance that have
instance variables are present in \df{type-bp-alist}, however.

This is a picture of the \df{cons-pair} type, as it actually appears
in memory:

\begin{center}
\begin{tabular}{|c|c|} \hline
 \multicolumn{2}{|c|}\emph{type} \\\hline
 \emph{instance-length:} & 3 \\\hline
 \emph{variable-length?:} & \texttt{\#f} \\\hline
 \emph{supertype-list:} & \texttt{(\emph{pair} \emph{object})}\\\hline
 \emph{ivar-list:} & \texttt{(the-car the-cdr)} \\\hline
 \emph{ivar-count:} & 2 \\\hline
 \emph{type-bp-alist:} & \texttt{((\emph{cons-pair} . 1))} \\\hline
 \emph{operation-method-alist:} & \texttt{((\emph{car} . \emph{meth}) $\ldots$)}\\\hline
 \emph{top-wired?:} & \texttt{\#f} \\\hline
\end{tabular}
\end{center}

\section{Storage Reclamation}

Our garbage collector \cite{PEARLMUTTER99} is a variant of Baker's
algorithm, a so-called ``stop and copy'' collector.  The spaces to be
reclaimed are renamed \emph{old}, all accessible objects in the old
spaces are transported to a new space, and the old spaces are
reclaimed.  The data present in the initial world is considered
``static'' and is not part of old space in normal garbage collections,
only in ``full'' garbage collections, which also move everything not
reclaimed into static space.  Due to locatives, the collector makes an
extra pass over the data; a paper with more complete details on this
latter complication is in press.  The weak pointer table is scanned at
the end of garbage collection, and references to deallocated objects
are discarded.

The user interface to the garbage collector is quite simple.
Normally, the user need not be concerned with storage reclamation;
upon the exhaustion of storage, the garbage collector is automatically
invoked.  When this happens some messages are printed; these messages
can be supressed with the \dfsw{-Q} switch.  The default size of new
space is 1Mb, or 256k references.  This can be altered with the
\dfsw{-h} \emph{size} switch, where \emph{size} is measured in bytes.  The
operations \df{\%gc} and \df{\%full-gc} invoke the garbage collector
explicitly.  Programs that use weak pointers can be effected by
garbage collection; for details, see section \ref{sec:weak}.

The \dfsw{-G} switch indicates that if and when the world is dumped, and
if Oaklisp terminates with an exit code of zero, a full garbage
collection should be performed.  In full garbage collections preceding
world dumps, the root set does not include the stacks.

New space is resized dynamically, being expanded to
\df{RECLAIM\protect\_FRACTION} times the amount of unreclaimed data if
the fraction of unreclaimed data is above more than one
\df{RECLAIM\protect\_FRACTION}'th of new space after a normal garbage
collection, or by the minimal amount needed if there is insufficient
space available in new space to fulfill the allocation request that
triggered the collector.  Currently \df{RECLAIM\protect\_FRACTION} is
two.  The \df{next\protect\_newspace\protect\_size} register says how
big the next new space allocated will be, and is accessible to Oaklisp
code.  Its value should not be lowered casually, as the garbage
collector will fail if new space is too small to hold all of the
non-reclaimed storage from old space.  A full garbage collection sets
the size of new space back to the value originally specified by the
user when Oaklisp was invoked, or the default value if none was
specified.
